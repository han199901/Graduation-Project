\chapter{需求分析及可行性分析}

\section{需求分析}
考试是检测个人能力的一个重要手段,而常见的考试方式除了传统的纸笔考试外,还有随着互联网一起快速发展的在线考试。
传统考试有自己的独特优点,而在线考试也不遑多让,在考试系统中很快就抢占了不少的份额。
在线考试系统自从桌面系统被普及后就已经在被使用,从原来的局域网考试,到互联网考试,再到现在的手机在线考试,
无论是C/S考试软件架构,还是B/S Web考试软件架构,在线考试系统一直都在不断发展,并在最近几年被广泛应用。
\par
在线考试快速发展的一个主要原因就是使用方便,考试成本低,节约时间。
同时在线考试系统能够对考生答卷中绝大多数题目智能阅卷并评分,大大减少了人力工作,正好适合学校平时练习和有考核需求的社会机构使用。
此外,在COVID-19大流行中,教育是唯一获得可观市场收入的行业。由于北美不同国家/地区的全国封锁,各种学校和大学都在远程进行考试。所有私立和公共教育机构均被迫中止课堂学习并适应在线学习。
\section{可行性分析}
经济可行性:随着经济技术的快速发展,目前硬件价格普遍下跌,宽带网大力建设,所以只需要在软件开发上面投入少许经费就可以了。
系统能降低管理费用和劳动费用,提高人员利用率,保证工作质量,人力资源合理分配,达到资源优化。这不仅给教师工作带来方便,
同时也满足了不同客户的不同需求,提高了数据的安全性、共享性,降低了预算,提高了工作效率,因此经济上可行。
\par
技术可行性:虽然本人在前后端开发经验上有所不足,但具有较强的动手能力,能够对老师所教导的内容进行融会贯通。
本次最大的挑战即时在系统中加入遗传算法并且平衡各方面相互制约的影响因素,使系统更加完善,我相信在老师的教导下能够快速且高质量的完成本次课题。
\par
管理可行性:目前,在线测试系统正在被更多的人所认可。在线测试系统能够实现无纸化考试,可以满足任何授权的考生随时随地考试并迅速获得成绩,
同时也大大减轻了教师出题和判卷等繁重的工作量。而且所面向的群体不止局限于学生,面向对象更广,为成年人创造了有利条件。


