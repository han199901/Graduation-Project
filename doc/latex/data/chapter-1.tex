\chapter{引言}

\section{课题的来源及意义}
在互联网飞速发展的今天,计算机和网络技术越来越广泛的应用于各个领域,改变着人们的学习、工作和生活。
随着教育现代化步伐的加快和计算机辅助教学的广泛应用,利用计算机的强大功能参与教学已成为教育工作者和教育科研人员广泛关注的研究领域。
目前,社会各个行业需要人才,而人才选拔的重要途径就是通过考试来判定。
现阶段,学校的考试大都是采用传统的考试方式:由老师出题,经复印后,学生纸上答题,老师人工阅卷,以及学生的学习状态分析也是人工进行的。
通常的出卷方式是参与教学的教师根据自己的知识、经验、风格来收集、选取并编制试题,这样做虽然试题的效率、信用度高,但同样存在着一定的缺点,主要表现在由于人为因素的不确定性,
可能会造成选题范围过于狭窄;耗费教师大量的时间、精力;不利于实现考、教分离。传统的考试方式由于工作量大从而容易出错。不仅如此,这样的考试方式不仅给老师带来了繁重的任务,
而且使得老师的工作效率大打折扣,不利于老师工作效率的提高。与传统的考试模式相比,在线考试具有无可比拟的优越性,具体优点如下:
\begin{enumerate}
	\item 将教师从繁重的出卷、阅卷、评卷的繁重工作中解脱出来,教师的工作效率大幅提高,减少人为主观意志对评分的影响,有效提高教育质量。
	\item 学生在平时的学习中,及时的在网上进行自我测试,在学习上能够查缺补漏,激发学生的学习兴趣,为学生的学习带来更多的方便。
	\item 系统实现自动组卷、客观题自动评分,使考试真正做到客观、公平、公正,真正实现考、教分离。
\end{enumerate}
\section{国内外的发展情况}
计算机考试系统的实现,将教师从繁琐的出题、监考、阅卷、试卷分析和成绩统计的传统考试中解脱出来,充分体现了准确、客观、公正、快速、简捷等特点。
20世纪70年代,美国考试委员会着手进行计算机模拟考试的研究工作,并于1983年编制出有效的模拟考试系统,当时的名称是计算机辅助考试系统。
1990年8月,美国加利福尼亚、得克萨斯等十个州创建各州以及各高等院校相互认可的学位证书以及相应的教学体系,从而正式拉开网络远程考试的序幕。
著名的考试机构有美国思而文学习系统有限公司。它是一家从事教育和计算机化考试服务的专业公司,在世界的6大洲140多个国家和地区有2200多个考试中心,
可用25种语言提供近百个不同类型,一千多种考试,每年全球参加计算机化考试的人数约400万。当今大部分的授证机构均委托思而文公司为其进行测试、评估。
最出名的网络教育案例,当属美国政府举办的TOFEL考试,目前在全球范围内,均可以通过国际互联网进行TOFEL培训与考试,大大减少了美国政府对于此项考试的开支,
并能更快速、准确地为期望进入美国学习的学生服务。目前美国约有 80 所大学允 许学生通过网络考试获得学位,另外,加拿大、英国等其它西方国家也在大力开展网络考试系统。
\par
与西方发达国家的突飞猛进相比,国内的计算机考试技术研究工作开展的相对较晚,但国内在网络远程教学研究工作发展相当迅速。
目前各高等院校如清华大学、北京 大学、上海复旦大学、同济大学、西安交通大学、华南理工大学、北京医科大学和湖南 大学等高校己陆续在网上设立了自己的考试系统,
并开展相关研究。国家信息产业部也开发了办公自动化证书CEAC远程考试系统、红旗 Linux 远程考试系统。通过对国内外计算机考试系统的考察和试用,我们发现它们具备以下特点:
\begin{itemize}
	\item C/S 结构和 B/S 结构并存,但基本都可以在网络上使用。
	\item 大都提供自动组卷和自动评卷功能,但水平参差不齐
	\item 考核软件使用仿真模拟环境和调用真实环境两者都有
	\item 均采 用了开放式试题库 , 扩充比较容易
	\item 对题库的分析管理部分都比较重 , 都提供了最基 本的功能
\end{itemize}
\section{课题开发的目标和内容}
\subsection{目标}
本课题设计是一个较为完善的在线的考试系统,并且拥有基本的用户管理。
\subsection{内容}
\begin{enumerate}
	\item 便捷建题:全面题型支持,客观题、主观题、自定义题型(A型题、阅读理解题、名词解释等);试题格式多样化,支持数学公式、化学分子式、图形图像、上下标、特殊符号全学科试题,完美解决数学公式、化学公式输出不失真;
	\item 题库管理:支持单人建设和多人共建一门题库,完善的辅助功能,严密的权限管理;
	\item 练习测试:学生课后练习或自学练习,老师可向学生开通一定比例的试题或手动挑选部分试题供学生练习,提供错题集和练习记录,理解和巩固相关知识点。提高学生练习的积极性,从而达到以练促学,以学促考的再次“翻转”;
	\item 组卷和排考:管理者基于试题库,支持组织试卷并通过安排考场、安排学生等简单流程即可完一场在线考试;
	\item 试卷评阅:客观题自动评阅,主观题由阅卷教师在线评阅,可根据题型逐题评阅,评阅时可分别针对试题给出评语;
	\item 教学分析:对学生成绩进行分析,形成评测报告;
	\item 监考功能:考生进入考试后,在考试过程中系统随机、自动抓拍五次考试照片,自动上传监控台存档;
\end{enumerate}




